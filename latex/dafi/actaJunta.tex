\documentclass[12pt,spanish,a4paper]{article}
\usepackage[spanish]{babel}
\usepackage[utf8]{inputenc}
\usepackage{lipsum}
\usepackage{soul}

\usepackage[
pdftex,
pdfauthor={DAETSIINF},
pdftitle={},
hidelinks]{hyperref}
\usepackage{bookmark}

% --- IMAGES ---
\usepackage{subfig}
\usepackage{graphicx}
\usepackage{float}
\DeclareGraphicsExtensions{.png,.jpg,.pdf,.mps,.gif,.bmp}
\usepackage{tikz}
\def\checkmark{\tikz\fill[scale=0.7](0,.35) -- (.25,0) -- (1,.7) -- (.25,.15) -- cycle;}
% --- IMAGES ---

% --- MARGIN DIMENSIONS ---
\frenchspacing \addtolength{\hoffset}{-1.5cm}
\addtolength{\textwidth}{3cm} \addtolength{\voffset}{-2.5cm}
\addtolength{\textheight}{4cm}
\setlength{\headheight}{15pt}
% --- MARGIN DIMENSIONS ---

% --- TOC DOTS ---
\usepackage[subfigure]{tocloft}
\renewcommand{\cftsecleader}{\cftdotfill{\cftdotsep}}
% --- TOC DOTS ---

% --- TITLE DATA ---
\title{
  \begin{flushright}
    \includegraphics[width=5cm]{include/daetsiinf-header.png}
  \end{flushright}
  \textbf{Junta de Delegados}\\
	\emph{DAETSIINF}}
\author{}
\date{\underline{\today}}
%\clearpage
% --- TITLE DATA ---

% --- DOCUMENT ---
\begin{document}
	% --- TITLE ---
	\maketitle
	\thispagestyle{empty}
	\pagenumbering{gobble}
	\renewcommand*\contentsname{Orden del día}
	\tableofcontents
	\pagebreak
	% --- TITLE ---
  \pagenumbering{arabic}
    \section*{Instrucciones a la hora de redactar}
      Según el número de puntos en el Orden del Día que haya, se eliminan o añaden secciones con su título entre llaves.
      En las que hay por defecto, hay que borrar el comando lipsum, y simplemente redactar \textst{\textsf{o copiar y pegar el texto si
      se redacta en otro sitio}}, recordando siempre que los saltos de línea se deben declarar con dos 'backslash' ($\backslash \backslash$) al
      final de la línea. \\
      Hay que distinguirlo del comando \verb=\par=, el cual nos separa dos textos en párrafos distintos, usando sangría de primera línea.\par
      \textit{Por cierto, estos dos párrafos también se borran.}
    \section{Aprobación, si procede, del Acta de la Junta anterior}
      \lipsum[1]
    \newpage
    \section{}
      \lipsum[2]
    \newpage
    \section{}
      \lipsum[3]
    \newpage
    \section{}
      \lipsum[4]
    \newpage
    \section{}
      \lipsum[5]
    \newpage
    \section{}
      \lipsum[6]
    \newpage
    \section{}
      \lipsum[7]
    \newpage
    \section{}
      \lipsum[8]
    \newpage
    \section{Ruegos y preguntas}
      \lipsum[9]
    \newpage
    \section{Resultados de las votaciones}
      \begin{center}
        \begin{tabular}{|c||c|c|c|}
          \hline
          Nº de apartado & Votos a favor & Votos en contra & Abstenciones \\
          \hline \hline
          Apartado 1 & & & \\
          \hline
          Apartado 2 & & & \\
          \hline
          Apartado 3 & & & \\
          \hline
          Apartado 4 & & & \\
          \hline
          Apartado 5 & & & \\
          \hline
          Apartado 6 & & & \\
          \hline
          Apartado 7 & & & \\
          \hline
          Apartado 8 & & & \\
          \hline
          Apartado 9 & & & \\
          \hline
        \end{tabular}
      \end{center}
    \section{Fin de la sesión}  
      Sin ningún otro turno de intervención solicitado, el Delegado de Alumnos del Centro levanta
      la sesión a las \textit{Introducir hora}.\par
      Doy fe como Secretario,

      \quad

      \quad

      \quad

      \begin{center}
        \textbf{Nombre del secretario}\\
        Secretaría de la Delegación de Alumnos\\
        ETSI Informáticos 
      \end{center}
\end{document}