\documentclass[12pt, a4paper, spanish]{article}
\usepackage[spanish]{babel}
\usepackage[utf8]{inputenc}
\usepackage{setspace}
\usepackage[official]{eurosym}
\usepackage{array}
\usepackage{float}
\usepackage[
pdftex,
pdfauthor={Nombre del secretario de la comisión},
pdftitle={Acta de Constitución de Ayudas a Asociaciones},
hidelinks]{hyperref}

% --- MATH OPERATIONS ---
\usepackage[nomessages]{fp}
% --- MATH OPERATIONS ---

% --- IMAGES ---
\usepackage{subfig}
\usepackage{graphicx}
\DeclareGraphicsExtensions{.png,.jpg,.pdf,.mps,.gif,.bmp}
% --- IMAGES ---

% --- MARGIN DIMENSIONS ---
\frenchspacing \addtolength{\hoffset}{-1.5cm}
\addtolength{\textwidth}{3cm} \addtolength{\voffset}{-2.5cm}
\addtolength{\textheight}{4cm}
\setlength{\headheight}{15pt}
% --- MARGIN DIMENSIONS ---

% --- TOC DOTS ---
\usepackage[subfigure]{tocloft}
\renewcommand{\cftsecleader}{\cftdotfill{\cftdotsep}}
% --- TOC DOTS ---

% --- TITLE DATA ---
\title{\textbf{Acta de Constitución de Ayudas a Asociaciones} \\[2ex]
	\emph{DAETSIINF}}
\author{\emph{Nombre del secretario de la comisión}}
\date{\underline{\today}}
% --- TITLE DATA ---

\makeatletter         
\def\@maketitle{
\raggedright
\begin{flushright}
\includegraphics[width = 6cm]{../include/daetsiinf-header.png}\\[8ex]
\end{flushright}
\begin{center}
{\Large \@title}\\ [4ex]
{\large  \@author}\\[4ex]
\@date\\[8ex]
\end{center}}
\makeatother


% --- DOCUMENT ---
\begin{document}
	% --- TITLE ---
	\maketitle
    \thispagestyle{empty}
	\pagenumbering{gobble}
	\renewcommand*\contentsname{Índice de contenidos}
	\tableofcontents
    \pagebreak
	% --- TITLE ---
	
	\pagenumbering{arabic}
    
    
    \section{Inicio de la sesión}
    En Boadilla del Monte, con fecha XX de XXXXXXX de XXXX , se procede a la constitución de la comisión de ayudas a asociaciones.\\
    A continuación se relacionan sus miembros, designados de acuerdo a la regulación aprobada.

    \section{Miembros presentes y cargos asignados}
    La relación de los miembros con sus cargos designados durante la constitución es la que sigue:

    \begin{itemize}
        \item \textbf{Presidente:} \underline{Apellido1 Apellido 2, Nombre}
        \item \textbf{Secretario:} \underline{Apellido1 Apellido 2, Nombre}
        \item \textbf{Vocales:} 
        \begin{itemize}
            \item \underline{Apellido1 Apellido 2, Nombre}
            \item \underline{Apellido1 Apellido 2, Nombre}
            \item \underline{Apellido1 Apellido 2, Nombre}
        \end{itemize}
        \item \textbf{Delegado de Alumnos:} \underline{Apellido1 Apellido 2, Nombre}
    \end{itemize}

    \quad

    \quad

    \quad 

    \quad

    \quad
    
    \quad
    
    \quad
    
    \quad
    
    \quad
    
    \quad

    \quad
    
    \quad

    \quad
    
    \quad

    \quad
    
    \quad

    \quad

    \begin{flushleft}
        Presidente \quad \quad Secretario \quad \quad Vocal 1 \quad \quad Vocal 2 \quad \quad Vocal 3 \quad \quad Delegado de Alumnos
        \begin{figure}[H]
            %\includegraphics[width=3cm]{include/firmaDani.png}
            %\includegraphics[width=3cm]{include/firmaArturo.png}
            %\quad \quad \includegraphics[width=3cm]{include/firmaSilvia.png}
            %\includegraphics[width=3cm]{include/firmaMoha.png}
            %\includegraphics[width=3cm]{include/firmaRevuelta.png}
        \end{figure}
    \end{flushleft}

    \newpage
    \section{Acuerdos tomados}
    \subsection{Asociacion X}
    \subsubsection{Proyectos}
    Descripción de los proyectos (se propone usar la lista de abajo):
        \begin{enumerate}
            % EJEMPLO DE PROYECTO CON VARIOS ELEMENTOS ASOCIADOS:
            \item Proyecto 1
                \begin{itemize}
                    \item Objeto 1 $\rightarrow Precio$ \euro{}
                    \item Objeto 2 $\rightarrow Precio$ \euro{}
                \end{itemize}
            
            % EJEMPLO DE PROYECTO CON UN SOLO ELEMENTO:
            \item Proyecto 2 $\rightarrow Precio$ \euro{}
            
            % AÑADIR MÁS COPIANDO Y PEGANDO LAS ESTRUCTURAS ANTERIORES!!!
        \end{enumerate}
    \subsubsection{Relación}
        \begin{center}
            % AÑADIR O QUITAR "|c" EN LAS LLAVES ABAJO PARA AÑADIR O QUITAR COLUMNAS
            \begin{tabular}{|c|c|c|}
                % CADA \HLINE IMPLICA UN BORDE HORIZONTAL ENTRE LÍNEAS
                \hline
                % SEPARAR LAS CELDAS CON & Y AÑADIR UN \\ AL FINAL DE LA LÍNEA
                Asociacion & Proyecto-1 & Proyecto-2 \\
                \hline
                % Rellenar las celdas vacías con la nota obtenida en cada área:
                Menores recursos económicos (40\%) & Nota & Nota \\
                \hline
                Idea innovadora (30\%) &  & \\
                \hline
                Participación (20\%) &  &  \\
                \hline
                Impacto (10\%) &  &  \\
                \hline \hline
                % Rellenar con las notas de las celdas anteriores multiplicadas por el porcentaje de cada área:
                \textbf{Total} & \textbf{Nota} & \textbf{Nota} \\
                \hline
            \end{tabular}
        \end{center}
    \newpage
    
    \subsection{TryIT!}
        \subsubsection{Proyectos}
        En esta categoría sólo se recibió un proyecto. Denominado \textbf{Nombre del proyecto} recibe como asignación la totalidad para esta categoría: \textbf{Precio \euro{}}.
    \newpage
    \subsection{Resolución final}
    Del total de Presupuesto \euro{} restantes presupuestados para las ayudas, se asignan de la siguiente manera acorde con las puntuaciones arriba desglosadas.
        \begin{center}
            \begin{tabular}{|c|c|c|}
                \hline
                \textbf{Posición} & \textbf{Proyecto} & \textbf{Dotación} \\
                \hline
                1 & Proyecto & Dinero asignado \euro{} \\
                \hline
            \end{tabular}
        \end{center}
    Queda por tanto abierto el plazo para presentación de reclamaciones hasta el día XX de XXXXXXX. \\
    \textbf{La resolución definitiva se hará el fecha.}
\end{document}